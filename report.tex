	\documentclass[notitlepage]{report}
	\renewcommand{\baselinestretch}{1.5}
	\author{Emil Centiu, zl18810}
	\title{Report \\
		\large{SPS Coursework: An Unknown Signal} \\
	}
	\begin{document}
		\maketitle
		
		\section*{general considerations}
		
		\section*{least squares}
		Because we can't assume anything about the random error term (i.e. follows a normal distribution like in MLE), Least squares was the go to method to find the best fitting line. 
		
		The goal is to minimise the sum of squared differences between the observed value of the dependent variable (yi) and the predicted value of the dependent variable (y hat), that is provided by the regression line.
		
		The best fit regression line must pass through the centroid.
		\section*{figures/plots}
		
		\section*{overfitting and new data}
				
		\section*{implementation}
		
	\end{document}